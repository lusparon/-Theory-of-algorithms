\documentclass[12pt,a4paper]{article}
\usepackage[T2A]{fontenc}
\usepackage[utf8]{inputenc}
\usepackage[english,russian]{babel}
\usepackage{amssymb}
\usepackage{amsfonts}
\usepackage{amsmath}
\usepackage{cmap}
\usepackage{indentfirst}
\usepackage{fancyhdr}
\usepackage{stmaryrd}
\usepackage{enumitem}
\usepackage{multicol}
\usepackage{tikz}
\usepackage{mathtools}

\textwidth=17cm
\voffset=-1cm
\hoffset=-0.5cm
\topmargin=0cm
\textheight=24.5cm
\oddsidemargin=0pt

\linespread{1.1}

\pagestyle{fancy}
%%%%%%%%%%%%%%%%%%%%%%%%%%%%%%%%%%%%%%%%%%%%%%%%%%%%%%%%%%%%%%%%%%%%%%%%%%%%%%%%%%%%%%%%%%%%%%%%%%%%%%%%%

\lhead{\bfseries Домашнее задание №1}
\rhead{Тызыхян Луспарон, 2.8}

%%%%%%%%%%%%%%%%%%%%%%%%%%%%%%%%%%%%%%%%%%%%%%%%%%%%%%%%%%%%%%%%%%%%%%%%%%%%%%%%%%%%%%%%%%%%%%%%%%%%%%%%% 


\fancyfoot{}

\renewcommand{\to}{\longrightarrow}
\newcommand{\tof}{\to\!\!.\ }
\newcommand{\tod}{\Longrightarrow}
\newcommand{\xtod}[1]{\xRightarrow{\text{#1\ } }}
\newcommand{\alphabet}[2]{#1=\{#2\}}
\newcommand{\salp}[1]{\alphabet{\Sigma}{#1}}
\AddEnumerateCounter{\asbuk}{\@asbuk}{\cyrm}

\begin{document}
\begin{center}
\section*{Домашнее задание №1}
\end{center}


\subsubsection*{Задания по упражнению 2}

\emph{Выполните один любой пункт упражнения.}

Решение(пункт а).
\begin{enumerate}
	\item Преобразование всех вхождений 1 на $ \varepsilon$ 
	\item Действие алгорифма заканчивает при любом входном слове
	\item Да , слова в которых нет ни одного вхлждения 1
\end{enumerate}

\subsubsection*{Задания по упражнению 3}

\emph{Выберите любые три схемы (1--6) и примените каждую к трём различным словам (а--м).}

Решение.
\begin{multicols}{3}
\begin{enumerate}
\item  $\begin{cases}
ab\to a,\\
b\to\varepsilon,\\
a\to b;
\end{cases}$%
\item $\begin{cases}
ba\to ab,\\
a\to\varepsilon,\\
b\tof b;
\end{cases}$
\item  $\begin{cases}
ab\to a,\\
b\tof\varepsilon,\\
a\to b;
\end{cases}$
\end{enumerate}
\end{multicols}

\setlength{\parindent}{0pt}
1a) $bba\underline{ab}\tod\underline{b}baa\tod\varepsilon\underline{b}aa\tod\varepsilon\varepsilon\underline{a}a\tod\varepsilon\varepsilon\underline{b}a\tod\varepsilon\varepsilon\varepsilon\underline{a}\tod\varepsilon\varepsilon\varepsilon\underline{b}\tod\varepsilon\varepsilon\varepsilon\varepsilon$ \\
1б)$a\underline{ab}bbaa\tod a\underline{ab}baa\tod a\underline{ab}aa\tod\underline{a}aaa\tod \underline{b}aaa\tod \varepsilon\underline{a}aa
\tod \varepsilon\underline{b}aa\tod \varepsilon\varepsilon\underline{a}a\tod \varepsilon\varepsilon\underline{b}a\tod \varepsilon\varepsilon\varepsilon\underline{a}\tod \varepsilon\varepsilon\varepsilon\underline{b}\tod \varepsilon\varepsilon\varepsilon\varepsilon$ \\
1в)$b\underline{ab}abab\tod ba\underline{ab}ab\tod baa\underline{ab}\tod \underline{b}aaa\tod \tod \varepsilon\underline{a}aa
\tod \varepsilon\underline{b}aa\tod \varepsilon\varepsilon\underline{a}a\tod \varepsilon\varepsilon\underline{b}a\tod \varepsilon\varepsilon\varepsilon\underline{a}\tod \varepsilon\varepsilon\varepsilon\underline{b}\tod \varepsilon\varepsilon\varepsilon\varepsilon$ \\
2a) $ b\underline{ba}ab\tod \underline{ba}bab\tod ab\underline{ba}b\tod a\underline{ba}bb\tod \underline{a}abbb\tod \varepsilon\underline{a}bbb\tod \varepsilon\varepsilon\underline{b}bb\tod \varepsilon\varepsilon bbb$\\
2б) $ aabb\underline{ba}a\tod aab\underline{ba}ba\tod aa\underline{ba}bba\tod aaabb\underline{ba}\tod aaab\underline{ba}b\tod aaa\underline{ba}bb\tod \underline{a}aaabbb\tod \varepsilon\underline{a}aabbb\tod \varepsilon\varepsilon\underline{a}abbb\tod \varepsilon\varepsilon\varepsilon\underline{a}bbb\tod \varepsilon\varepsilon\varepsilon\varepsilon\underline{b}bb\tod \varepsilon\varepsilon\varepsilon\varepsilon bbb$\\ 
2в) $ \underline{ba}babab\tod ab\underline{ba}bab\tod a\underline{ba}bbab\tod aabb\underline{ba}b\tod aab\underline{ba}bb\tod aa\underline{ba}bbb\tod \underline{a}aabbbb\tod \varepsilon\underline{a}abbbb\tod \varepsilon\varepsilon\underline{a}bbbb\tod \varepsilon\varepsilon\varepsilon\underline{b}bbb\tod \varepsilon\varepsilon\varepsilon bbbb$\\


\subsubsection*{Задания по упражнению 4}

\emph{Выполните любые шесть заданий (кроме б, д, м).}

Решение.


\end{document}
