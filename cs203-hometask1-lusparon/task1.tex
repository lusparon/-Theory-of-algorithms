\documentclass[12pt,a4paper]{article}
\usepackage[T2A]{fontenc}
\usepackage[utf8]{inputenc}
\usepackage[english,russian]{babel}
\usepackage{amssymb}
\usepackage{amsfonts}
\usepackage{amsmath}
\usepackage{cmap}
\usepackage{fancyhdr}
\usepackage{enumitem}
\usepackage{multicol}
\usepackage{url}
\usepackage[center]{titlesec}
\usepackage{mathtools}
\textwidth=17cm
\voffset=-1cm
\hoffset=-0.5cm
\topmargin=0cm
\textheight=24.5cm
\oddsidemargin=0pt

\linespread{1.1}

\pagestyle{fancy}
\lhead{\bfseries CS203. Теория алгоритмов}
\rhead{\itshape 2017/2018 (весенний семестр)}
\fancyfoot{}

\renewcommand{\theenumii}{\asbuk{enumii}}
\AddEnumerateCounter{\asbuk}{\@asbuk}{\cyrm}

\renewcommand{\to}{\longrightarrow}
\newcommand{\tof}{\to\!\!.\ }
\newcommand{\tod}{\Longrightarrow}
\newcommand{\xtod}[1]{\xRightarrow{\text{#1\ } }}
\newcommand{\alphabet}[2]{#1=\{#2\}}
\newcommand{\salp}[1]{\alphabet{\Sigma}{#1}}

\begin{document}
\section*{Задание №1: нормальные алгорифмы А.\,А.\,Маркова~(младшего)}

\begin{enumerate}

\item Дана схема нормального алгорифма в алфавите $\salp{a, b}$:
\[
  \begin{cases}a\tof\varepsilon,\\ b\to bb. \end{cases}
\]
\begin{enumerate}
\item Определите смысл действия этого алгорифма.
\item Приведите примеры слов, на которых действие алгорифма завершается.
\item Приведите примеры слов, на которых действие алгорифма не завершается.
\item Имеются ли слова, на которых алгорифм завершается из-за того, что к текущему слову неприменима ни одна из подстановок?
\end{enumerate}

\item Ответьте на те же вопросы, что и в первом упражнении, если схема нормального алгорифма в алфавите $\salp{1}$ состоит из одной подстановки:
%  \begin{multicols}{2}
  \begin{enumerate}
  \item  $1\to\varepsilon$;
  \item $\varepsilon\tof 1$.
  \end{enumerate}
%  \end{multicols}

\item Даны схемы нормальных алгорифмов в алфавите $\salp{a,b}$:
\begin{multicols}{3}
  \begin{enumerate}[label=\arabic*)]
\item  $\begin{cases}
  ab\to a,\\
  b\to\varepsilon,\\
  a\to b;
\end{cases}$%
\item $\begin{cases}
  ba\to ab,\\
  a\to\varepsilon,\\
  b\tof b;
\end{cases}$
\item  $\begin{cases}
  ab\to a,\\
  b\tof\varepsilon,\\
  a\to b;
\end{cases}$
\item 
$\begin{cases}
  ab\to b,\\
  ba\to bb,\\
  b\tof \varepsilon;
\end{cases}$
\item $\begin{cases}
ba\to a,\\
bb\to b,\\
ab\to\varepsilon,\\
\varepsilon\tof b;  
\end{cases}$
\item $\begin{cases}
bb\to ba,\\
ba\to a,\\
a\to\varepsilon,\\
\varepsilon\tof b.
\end{cases}$
\end{enumerate}
 \end{multicols}

Примените эти алгорифмы к следующим словам:
\begin{multicols}{3}
\begin{enumerate}
\item $bbaab$;
\item $aabbbaa$;
\item $bababab$;
\item $aaaa$;
\item $bbbbb$;
\item $aabaabb$;
\item $abbbbba$;
\item $baab$;
\item $bbbaaa$;
\item $abbabba$;
\item $abbbaaab$;
\item $\varepsilon$.
\end{enumerate}
\end{multicols}
\newpage

\item Напишите схемы нормальных алгорифмов для решения следующих задач:
\begin{enumerate}[topsep=0pt,partopsep=0pt,itemsep=1pt]
\item Преобразование любого слова в алфавите $\salp{a,b,c}$ в $\varepsilon$.

\item Проверка чётности числа, заданного в унарной системе счисления (в алфавите $\salp{1}$).
Число должно преобразовываться в $1$, если оно нечётное, и~в~$\varepsilon$ в~противном случае.

\item Увеличение на единицу числа, заданного в десятичной записи.

\item Увеличение числа, заданного двоичной записью, на 2.

\item Приписывание к концу любого входного слова $abab$ (над алфавитом $\salp{a,b}$).

\item Приписывание к концу любого непустого входного слова символа $a$, если в слове
 присутствует чётное количество символов $a$, и символа $b$ в противном случае (над алфавитом $\salp{a,b}$).

\item Вычисление частного от деления числа, заданного в унарной системе счисления, на 2.

\item Умножение числа, заданного в унарной системе счисления, на 2.

\item Вычисление частного и остатка от деления числа, заданного в унарной системе счисления, на 
два (над алфавитом $\salp{1,\#}$). Результат должен записываться в виде «частное$\#$остаток». 
Ноль должен соответствовать пустому слову.

\item Дублирование всех символов входного слова (над алфавитом $\salp{a,b}$). 
Например: $abab\Rightarrow aabbaabb$.

\item Перестановка символов входного слова в обратном порядке (над $\salp{a,b}$).

\item Cортировка символов входного слова (над алфавитом $\salp{a,b,c}$).

\item Проверка, является ли входное слово палиндромом (над алфавитом $\salp{a,b}$). 
Если является, то результатом должно быть пустое слово,
если не является, то результатом может быть любое непустое слово.

\item Проверка, является ли входное слово именем одного из основных регистров процессора Intel 8088
(AX, BX, CX или DX).  Результатом должно быть либо имя регистра, либо пустое слово.
\end{enumerate}

\end{enumerate}

\subsubsection*{Задание на программирование (до 5 бонусных баллов)}

 Напишите на любом языке программирования программу, которая принимает на вход схему нормального алгорифма в произвольном формате вместе с входным словом и печатает результат применения схемы к слову вместе со всеми промежуточными преобразованиями. \emph{Указание.} Это задание должно выполняться в собственном закрытом репозитории на github.com, который можно получить в рамках студенческого пакета для разработчиков (\url{https://education.github.com/pack}). Для начисления бонусных баллов задание на программирование должно быть выполнено до конца семестра.

\end{document}

%%% Local Variables:
%%% mode: latex
%%% TeX-master: t
%%% End:
