\documentclass[12pt,a4paper]{article}
\usepackage[T2A]{fontenc}
\usepackage[utf8]{inputenc}
\usepackage[english,russian]{babel}
\usepackage{amssymb}
\usepackage{amsfonts}
\usepackage{amsmath}
\usepackage{tikz}
\usepackage{cmap}
\usepackage{fancyhdr}
\usepackage{enumitem}
\usepackage{nicefrac}
\usepackage{multicol}
\usepackage[center]{titlesec}

\textwidth=18cm
\hoffset=-1cm
\voffset=-1cm
%\headsep=0cm
\topmargin=0cm
\textheight=23.9cm
\oddsidemargin=0pt
\parindent=1.25cm

\linespread{1}

\pagestyle{empty}
\pagestyle{fancy}
\lhead{\bfseries Теория алгоритмов}
\rhead{\itshape 2018/2019 (весенний семестр)}
\fancyfoot{}


\renewcommand{\theenumii}{\asbuk{enumii}}
\AddEnumerateCounter{\asbuk}{\@asbuk}{\cyrm}

\newcommand{\monus}{\stackrel{{}^{\scriptstyle .}}{\smash{-}}}
\newcommand{\recop}[2]{\textbf{#1}(#2)}
\newcommand{\Comp}[1]{\recop{comp}{#1}}
\newcommand{\Mn}[1]{\recop{Mn}{#1}}
\newcommand{\Prr}[1]{\recop{prim}{#1}}

\input{LambdaDefs}


\begin{document}
\section*{Задания №4 и №5: регистровые машины,\\ примитивная рекурсивность, кодирование}

\begin{enumerate}[itemsep=3pt]

\item Написать регистровую машину, которая копирует значение входного регистра $x_2$ 
в выходной регистр $x_1$. Значение входного регистра в конце работы поменяться не должно.

\item Написать регистровые машины для вычисления следующих функций:
  \begin{enumerate}
  \begin{multicols}{2}
    \item $\operatorname{sg}x=\begin{cases}0, \mbox{ если }x=0,\\ 1,
        \mbox{ если }x>0;\end{cases}$
    \item $f(x,y)=x+ y$;
    \item $f(x,y)=|x-y|$;
    \item $f(x,y)=x \operatorname{div} y$;
    \item $f(x,y)=x \operatorname{mod} y$.
  \end{multicols}
  \end{enumerate}

\item Вычислить значения следующих выражений (\textbf{comp} --- оператор композиции, \textbf{prim}~---~оператор примитивной рекурсии):
  \begin{multicols}{2}
    \begin{enumerate}
    \item $\Comp{P^2_2, Z,S}(9)$;
    \item $\Prr{P^2_2,\Comp{S, P^4_4}}(6,35,2)$;
    \item $\Comp{\cdot, P^3_2,P^3_2}(7, 14, 8)$;
    \item $\Prr{S,h}(7,2)$, где $h(x,y,z) = x+y+z$.
    \end{enumerate}
  \end{multicols}
\item Доказать, что следующие функции являются примитивно рекурсивными, построив для них
схему примитивной рекурсии:
  \begin{multicols}{2}
    \begin{enumerate}
    \item $\varphi(x,y)=x + y$;
    \item $\varphi(x,y)=x\cdot y$;
    \item $\varphi(x)=x!$ (здесь $0!=1$);
    \item $\varphi(x,y)=x^y$ (здесь $0^0=1$);
    \item $\mathrm{sg}(x)=\begin{cases}0, \mbox{ если }x=0,\\ 1,
        \mbox{ если }x>0;\end{cases}$
    \item $\overline{\mathrm{sg}}(x)=\begin{cases}1, \mbox{ если
        }x=0,\\ 0, \mbox{ если }x>0.\end{cases}$
    \end{enumerate}
  \end{multicols}
Записать эти функции в виде из предыдущего упражнения.

\item Доказать, что функция $f$, определенная следующим образом, является примитивно рекурсивной:
\[
\begin{array}{l}
f(0)=1;\quad f(1)=2;\quad  f(2)=3;\quad f(n)=0, \mbox{\ для $n>2$}.
\end{array}
\]
\item Доказать, что следующие функции примитивно рекурсивны, представив их в виде композиции
  сложения и усечённой разности ($x\monus y$):
  \begin{multicols}{3}
    \begin{enumerate}
    \item $|x-y|$;
    \item $\min(x,y)$;
    \item $\max(x,y)$.
    \end{enumerate}
  \end{multicols}



\item Доказать, что множество $\mathcal{PR}$ замкнуто относительно следующих операций, 
выразив их через оператор композиции:
  \begin{enumerate}
  \item получение $\lambda \bar x \bar z.f(\bar x)$ из $\lambda \bar x.f(\bar x)$ (введение неиспользуемых переменных);
  \item получение $\lambda \bar x y \bar z w \bar u.f(\bar x,w,\bar z,y,\bar u)$ из 
$\lambda \bar x y \bar z w \bar u.f(\bar x,y,\bar z,w,\bar u)$ (перестановка переменных);
\item получение $\lambda \bar x y \bar z \bar u.f(\bar x,y,\bar z,y,\bar u)$ из 
$\lambda \bar x y \bar z w \bar u.f(\bar x,y,\bar z,w,\bar u)$ (отождествление двух переменных).
  \end{enumerate}


\item Будем называть функцию чётной, если её значение на чётных аргументах всегда чётно. 
Проверить, сохраняется ли свойство чётности для композиции чётных функций.
  
\item Доказать, что следующие функции и предикаты являются примитивно рекурсивными:
  \begin{enumerate}
  \item $\tau(x)$ — число делителей $x$, $\tau(0)=0$;
  \item $\sigma(x)$ — сумма делителей $x$, $\sigma(0)=0$;
  \item $\tau_p(x)$ — число простых делителей $x$, $\tau_p(0)=0$;
  \item $D_{m,n}(x)$ — $x$ делится на $m$ и на $n$;
  \item $E(x)$ --- $x$ чётно;
  \item $N(x,y)\equiv x\neq y$;
  \item $\mathrm{lcm}(x,y)$ --- наименьшее общее кратное чисел $x$ и $y$;
  \item $\mathrm{gcd}(x,y)$ --- наибольший общий делитель чисел $x$ и $y$;
  \item $\mathrm{MPr}(x,y)$ --- $x$ и $y$ взаимно просты;
  \item $\mathrm{sqrt}(x) = \lfloor\sqrt x\rfloor$ --- целая часть числа $\sqrt x$;
  \item $\mathrm{rt}(x, y) = \lfloor\sqrt[y] x\rfloor$ --- целая часть числа $\sqrt[y] x$,
где $\sqrt[0]x=x$;
  \item $\phi(x) = \lfloor x\sqrt 2\rfloor$ --- целая часть числа $x \sqrt 2$.
  \end{enumerate}


\item Закодировать следующие числовые последовательности:
  \begin{enumerate}
    \begin{multicols}{4}
  \item  $[6,3,2]$;
  \item   $[1,5,2]$; 
  \item   $[7,4,1,2]$;
  \item   $[3,2,1,2]$.
    \end{multicols}
  \end{enumerate}

\item Раскодировать числовые последовательности с кодами:
  \begin{enumerate}
    \begin{multicols}{4}
  \item 97200;
  \item 72000;
  \item 102900;
  \item 1131900.
    \end{multicols}
  \end{enumerate}

\item Нормализовать, реализовать и закодировать следующие регистровые машины 
вместе с их завершающимися вычислениями (для заданных значений входных регистров):
  \begin{enumerate}
    \begin{multicols}{2}
    \item $M_x=\Lx 2$;
    \item $M^x_y=\Lx \mathrm{sg}(x)$, $x=0$;
    \item $M^x_y=\Lx \overline{\mathrm{sg}}(x)$, $x=0$;
    \item $M^x_y=\Lx x + 1$, $x=1$;
    \item $M^x_y=\Lx x\monus 1$, $x=1$;
    \item $M^{xy}_z=\Lvar{xy}x+y$, $x=2$, $y=1$.
    \end{multicols}
  \end{enumerate}

\item Раскодировать регистровые машины с кодами:
  \begin{enumerate}
    \begin{multicols}{2}
    \item $2^{6301}3^{1729}$;
    \item $2^{308701}3^{26575698996001}5^{16201}7^{15553}$.
    \end{multicols}
  \end{enumerate}


\item Пользуясь неограниченным поиском, доказать утверждение: 
если график $z=f(\bar x)\in{\mathcal R_*}$ и $f$ всюду определена, 
то $f\in\mathcal R$.

\item Доказать индукцией по $n$, что для всех $n$ имеет место неравенство $p_n\leqslant 2^{2^n}$,
где $p_n$ --- $n$-е простое число, считая с нуля. 
\emph{Указание}. Рассмотреть число 
$p_0p_1\cdots p_n+1$ и ответить на вопросы: простое оно или составное, 
как оно соотносится с $p_{n+1}$, каковы его возможные делители.


\item а) Доказать, что $P(\bar x_n)\in\mathcal P_*$ тогда и только тогда, когда
существует частично рекурсивная функция $\Lvar{\bar x_n} g(\bar x_n)$ такая, что 
для всех $\bar x_n$ отношение $P(\bar x_n)\equiv g(\bar x_n)=0$. 

б) Пользуясь этим фактом, доказать, что $\mathcal{PR}_*\subset\mathcal P_*$ и 
$\mathcal{R}_*\subset\mathcal P_*$.

\end{enumerate}


\end{document}

%%% Local Variables:
%%% mode: latex
%%% TeX-master: t
%%% End:
