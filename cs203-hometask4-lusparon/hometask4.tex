\documentclass[12pt,a4paper]{article}
\usepackage[T2A]{fontenc}
\usepackage[utf8]{inputenc}
\usepackage[english,russian]{babel}
\usepackage{amssymb}
\usepackage{amsfonts}
\usepackage{amsmath}
\usepackage{cmap}
\usepackage{indentfirst}
\usepackage{fancyhdr}
\usepackage{stmaryrd}
\usepackage{enumitem}
\usepackage{multicol}
\usepackage[center]{titlesec}
\usepackage{tikz}
\usetikzlibrary{arrows,positioning,shapes,automata} 

\textwidth=17cm
\voffset=-1cm
\hoffset=-0.5cm
\topmargin=0cm
\textheight=24.5cm
\oddsidemargin=0pt

\pagestyle{fancy}
%%%%%%%%%%%%%%%%%%%%%%%%%%%%%%%%%%%%%%%%%%%%%%%%%%%%%%%%%%%%%%%%%%%%%%%%%%%%%%%%%%%%%%%%%%%%%%%%%%%%%%%%%

\lhead{\bfseries Домашнее задание №4}
\rhead{Тызыхян Луспарон, 2.8}

%%%%%%%%%%%%%%%%%%%%%%%%%%%%%%%%%%%%%%%%%%%%%%%%%%%%%%%%%%%%%%%%%%%%%%%%%%%%%%%%%%%%%%%%%%%%%%%%%%%%%%%%% 


\fancyfoot{}
\renewcommand{\L}{\ensuremath{\lambda}}
\newcommand{\Lvar}[1]{\ensuremath{\L #1\,.\,}}
\newcommand{\Lx}{\Lvar{x}}
\newcommand{\Ly}{\Lvar{y}}
\newcommand{\Lz}{\Lvar{z}}
\newcommand{\Lt}{\Lvar{t}}
\newcommand{\Ln}{\Lvar{n}}



% Команды для описания регистровых машин

\newcommand{\monus}{\stackrel{{}^{\scriptstyle .}}{\smash{-}}}
\newcommand{\recop}[2]{\textbf{#1}(#2)}
\newcommand{\Comp}[1]{\recop{comp}{#1}}
\newcommand{\Prr}[1]{\recop{prim}{#1}}
\newcommand{\uif}[3]{\mathrm{if}\ #1\  \mathrm{then\ goto}\ #2\ \mathrm{else\ goto}\ #3}
\newcommand{\ustop}{\mathrm{stop}}
\newcommand{\uleft}{\leftarrow}


\newcommand{\PR}{\mathcal{PR}}

\begin{document}
\section*{Домашнее задание №4}


\begin{enumerate}
\item Упражнение 1.\\
$M_{x_1}^{x_2}\\
1: \uif{x_2=0}{6}{2}\\
2: x_1\uleft x_1+1\\
3: x_2\uleft x_2\monus 1\\
4: x_3\uleft x_3+1\\
5: goto\;1\\
6: \uif{x_3=0}{10}{7}\\
7: x_2\uleft x_2+1\\
8: x_3\uleft x_3\monus 1\\
9: goto\;6\\
10:\ustop\\$

\item Упражнение 2 (любые три пункта).\\
(a) $M_{x_1}^{x_2}\\
1: \uif{x_2=0}{3}{2}\\
2: x_1\uleft 1\\
3: stop\\$\\
(б) $M_{c}^{xy}\\
1: c\uleft x\\
2: i\uleft y\\
3: \uif{i=0}{7}{4}\\
4: c\uleft c+1\\
5: i\uleft i\monus 1\\
6: goto\;3\\
7: stop\\ $\\
(г) $M_{c}^{xy}\\
1: \uif{x=0}{11}{2}\\
2: i\uleft y\\
3: i\uleft i\monus 1\\
4: \uif{i=0}{9}{5}\\
5: \uif{x=0}{11}{6}\\
6: x\uleft x\monus 1\\
7: i\uleft i\monus 1\\
8: goto\;4\\
9: c\uleft c+1\\
10: goto\;1\\
11: stop\\$

\item Упражнение 3 (в, г).\\
(в)$comp(\cdot, P_2^3, P_2^3)(7, 14, 8) = P_2^3(7, 14, 8) \cdot P_2^3(7,14,8) = 14 \cdot 14 = 196$\\
(г)$prim(S,h)(7,2) = f(7,2) = \\
= 6 + 2 + f(6,2) = 8 + 7 + f(5,2) = 15 + 6 + f(4,2) = \\
= 21 + 5 + f(3,2) = 26 + 4 + f(2,2) = 30 + 3 + f(1,2) = \\
= 33 + 2 + f(0,2) = 35 + 3 = 38 $\\
\item Упражнение 4 (в, е).\\
(в) $\varphi(x) = x!\;$(здесь $0! = 1)\\
\varphi(x)=\widetilde p(x,y)\\
\widetilde p(0,y)=1=S(Z(y))=comp[S,Z](y)\\
\widetilde p(x+1,y)=(x+1) \cdot \widetilde p(x,y) =S(x) \cdot \widetilde p(x,y) = comp[\cdot, comp[S, P_1^3], P_3^3](x,y,\widetilde p(x,y))\\
\varphi(x)=prim[comp(S,Z),comp[\cdot, P_1^3, P_3^3]]\\$
(е) $\overline{\mathrm{sg}}(x)=\begin{cases}1, \mbox{ если
}x=0,\\ 0, \mbox{ если }x>0.\end{cases}\\
\overline{\mathrm{sg}}(x) = \widetilde p(x,y)\\ 
\widetilde p(0,y)=1=S(Z(y))=comp[S,Z](y)\\
\widetilde p(x+1,y)=0=P_2^2(x,Z(y))\\
\overline{\mathrm{sg}}(x) = prim[comp(S,Z),P_2^2(x,Z(y))]$\\
\item Упражнение 6 (любые два пункта).\\
(a) $\varphi(x,y)=|x-y|\\
\varphi(0,y)=y=P_1^1(y)\\
\varphi(x+1,y)=(x\monus y)+(y\monus x)=comp[+,comp[\monus,P_1^3,P_2^3], comp[\monus,P_2^3,P_1^3]](x,y,\varphi(x,y))\\
|x-y|=prim[P_1^1,comp[+,comp[\monus,P_1^3,P_2^3],comp[\monus,P_2^3,P_1^3]]]$

(в) $\varphi(x,y)=max(x,y)\\
\varphi(0,y)=y=P_1^1(y)\\
\varphi(x+1,y)=x + (y\monus x)=comp[+,P_1^2, comp[\monus, P_1^2, P_2^2]] (x,y)\\
max(x,y)=prim[P_1^1,comp[+,P_1^2, comp[\monus, P_1^2, P_2^2]]]\\$

\item Упражнение 7 (один любой пункт) — символы надчёркивания над именами переменных можно игнорировать.\\
(в)$
\lambda xyzu.f(x,y,z,y,u)=comp[f,P_1^5,P_2^5,P_3^5,P_2^5,P_5^5](x,y,z,w,u)
\\$
\item Задание на 4 бонусных балла: упражнения 5 и 8.\\
Упражнение 5.\\
$f(0)=1; f(1)=2; f(2)=3; f(n)=0, $ для $ n>2.\\
f(x) = \widetilde p(x,y) \\
\widetilde p(0,y)=1=S(Z(y))=comp[S,Z](y)\\
\widetilde p(x+1,y)=((x+1)\monus 1)\cdot(3\monus (x+1)) + (3\monus (x+1))\cdot \widetilde p(x,y)
=comp[+, comp[\cdot, comp[\monus, comp[+, P_1^3, 1], 1], comp[\monus, 3, comp[+, P_1^3, 1]]], comp[\cdot, comp[\monus, 3, comp[+, P_1^3,1]],P_3^3](x,y,\widetilde p(x,y))\\
\widetilde p=prim[comp[S,Z], comp[+, comp[\cdot, comp[\monus, comp[+, P_1^3, 1], 1], comp[\monus, 3, comp[+, P_1^3, 1]]], comp[\cdot, comp[\monus, 3, comp[+, P_1^3,1]],P_3^3]
\\$

Упражнение 8.\\
Из определения композиции слудует , что в аргумент первой функции мы подаем возвращаемое значение второй функции , которое всегда будет четным.Следовательно аргумент первой функции четный и отсюда возвращаемое значение тоже будет чётным.ЧТД.\
\end{enumerate}

\end{document}
