\documentclass[12pt,a4paper]{article}
\usepackage[T2A]{fontenc}
\usepackage[utf8]{inputenc}
\usepackage[english,russian]{babel}
\usepackage{amssymb}
\usepackage{amsfonts}
\usepackage{amsmath}
\usepackage{cmap}
\usepackage{indentfirst}
\usepackage{fancyhdr}
\usepackage{stmaryrd}
\usepackage{enumitem}
\usepackage{multicol}
\usepackage[center]{titlesec}
\usepackage{tikz}
\usetikzlibrary{arrows,positioning,shapes,automata} 
\tikzset{
	treenode/.style = {align=center, inner sep=0pt, text centered,
		font=\sffamily,minimum height=1.2em},
	app/.style = {treenode, black, rectangle, font=\itshape,
		text width=2.2em},% arbre rouge noir, noeud noir
	lam/.style = {treenode,  black,  
		text width=1.5em},% arbre rouge noir, noeud rouge
	var/.style = {treenode, circle, 
		minimum width=1.5em, minimum height=0.5em}% arbre rouge noir, nil
}

\textwidth=17cm
\voffset=-1cm
\hoffset=-0.5cm
\topmargin=0cm
\textheight=24.5cm
\oddsidemargin=0pt

\pagestyle{fancy}
%%%%%%%%%%%%%%%%%%%%%%%%%%%%%%%%%%%%%%%%%%%%%%%%%%%%%%%%%%%%%%%%%%%%%%%%%%%%%%%%%%%%%%%%%%%%%%%%%%%%%%%%%

\lhead{\bfseries Домашнее задание №5}
\rhead{Тызыхян Луспарон, 2.8}

%%%%%%%%%%%%%%%%%%%%%%%%%%%%%%%%%%%%%%%%%%%%%%%%%%%%%%%%%%%%%%%%%%%%%%%%%%%%%%%%%%%%%%%%%%%%%%%%%%%%%%%%% 


\fancyfoot{}
\renewcommand{\L}{\ensuremath{\lambda}}
\newcommand{\Lvar}[1]{\ensuremath{\L #1\,.\,}}
\newcommand{\Lx}{\Lvar{x}}
\newcommand{\Ly}{\Lvar{y}}
\newcommand{\Lz}{\Lvar{z}}
\newcommand{\Lt}{\Lvar{t}}
\newcommand{\Ln}{\Lvar{n}}

\newcommand{\betared}{\rightarrow_\beta}
\newcommand{\betareds}{\rightarrow_\beta^*}

\newcommand{\Lif}[3]{\text{if } #1 \text{ then } #2 \text{ else } #3}
%\newcommand{\Lif}[3]{\mathrm{if}\ #1\ \mathrm{then}\ #2\ \mathrm{else}\ #3}
\newcommand{\letin}[3]{\mathrm{let}\ #1 = #2\ \mathrm{in}\ #3}

\newcommand{\isZero}[1]{\text{\ttfamily is\_zero? } #1}
\newcommand{\Lexp}{\text{\ttfamily{} exp }}
\newcommand{\Lplus}{\text{\ttfamily plus }}
\newcommand{\Lsucc}{\text{\ttfamily succ }}
\newcommand{\Lpred}{\text{\ttfamily pred }}
\newcommand{\LNat}{\text{Nat}}
\newcommand{\LBool}{\text{Bool}}

\newcommand{\false}{\text{\ttfamily false }}
\newcommand{\true}{\text{\ttfamily true }}
\newcommand{\Land}{\text{\ttfamily{} and }}
\newcommand{\Lor}{\text{\ttfamily{} or }}
\newcommand{\Lnot}{\text{\ttfamily not }}

\newcommand{\fst}{\text{\ttfamily fst }}
\newcommand{\snd}{\text{\ttfamily snd }}

\DeclareMathOperator{\FV}{FV}

\newcommand{\Ycr}{\mathbf Y_{\text{CR}}}
\newcommand{\Yt}{\mathbf Y_{\text{Turing}}}

\newcommand{\clS}{\mathbf S}
\newcommand{\clK}{\mathbf K}
\newcommand{\clB}{\mathbf B}
\newcommand{\clI}{\mathbf I}
\newcommand{\clW}{\mathbf W}
\newcommand{\clC}{\mathbf C}
\newcommand{\clY}{\mathbf Y}
\newcommand{\CLvar}[1]{\ensuremath{[#1] \, . \,}}

\newcommand{\cI}{\textbf{I}}
\newcommand{\cK}{\textbf{K}}
\newcommand{\cS}{\textbf{S}}





\begin{document}
\section*{Домашнее задание №5}

\subsection*{Упражнения по заданию №5}

\begin{enumerate}
\item Упражнение 9 (любые 6 пунктов).\\
\item Упражнение 10 (любые два пункта).\\
a)
$[6,3,2] = 2^7 * 3^4 * 5^3 = 128 * 81 * 125 = 1296000 $

б)
$[1,5,2] = 2^2 * 3^6 * 5^3 = 4 * 729 * 125 = 364500 $

\item Упражнение 11 (любые два пункта).\\
a)
$97200 = 2^4 * 3^5 * 5^2 $\\
$2^4 * 3^5 * 5^2 = [3,4,1]$
\newline

б)
$72000 = 2^6 * 3^2 * 5^3 $\\
$2^6 * 3^2 * 5^3 = [5,1,2]$

\item Упражнение 12 (любые три пункта).\\
a)
$M_x=\Lx 2$

\text{Нормализация:} $M_{x_{1}}$\\
\text{Реализация: }
\begin{flalign*}
1&: x_1 \uleft 2&&\\
2&: \ustop&&
\end{flalign*}

\text{Кодирование: } \\
$1)[1,1,1,2]$\\
$2)[5,2]$\\


$1) 2^2*3^2*5^2*7^3 = 308700$\\
$2) 2^6 * 3^3 = 1728$

\text{Код машины : } $2^{308701} * 3^{1729}$\\
\newline

в)
$M^x_y=\Lx \overline{\mathrm{sg}}(x)$, $x=0$

\text{Нормализация:} $M_{x_{1}}^{x_{11}}$\\
\text{Реализация: }
\begin{flalign*}
1&: \uif{x_{11}=0}{2}{3}&&\\
2&: x_{1}\uleft 1&&\\
3&: \ustop&&
\end{flalign*}

\text{Кодирование: } \\
$1)[4,1,2,2,3]$\\
$2) [1,2,1,1]$\\
$3)[5,3]$\\


$1) 2^5*3^2*5^3*7^3*11^4 = 180787068000$\\
$2) 2^2*3^3*5^2*7^2 = 132300$\\
$3) 2^6*3^4 = 5184$

\text{Код машины : } $2^{180787068001} * 3^{132301} * 5^{5185}$\\
\text{при x=0 :}\\
$1;0,0$\\
$2;0,1$\\
$3;0,1$\\

\text{Кодировка завершающегося вычисления: }\\

$[1,0,0] = 2^2*3^1*5^1 = 60$\\
$[2,0,1]= 2^3*3^1*5^2=600$\\
$[3,0,1] = 2^4*3^1*5^2= 1200$\\

\text{Код вычисления :} $2^{61} * 6^{601} * 5^{1201}$\\
\newline

г)
$M^x_y=\Lx x + 1$, $x=1$

\text{Нормализация:} $M_{x_{1}}^{x_{11}}$\\
\text{Реализация: }
\begin{flalign*}
1&: \uif{x_{11}=0}{5}{2}&&\\
2&: x_{1}\uleft x_{1} + 1&&\\
3&: x_{11}\uleft x_{11} \monus 1&&\\
4&: \uif{x_{11}=0}{5}{1}&&\\
5&: x_{1}\uleft x_{1} + 1&&\\
6&: \ustop&&
\end{flalign*}

\text{Кодирование: } \\
$1)[4,1,2,5,2]$\\
$2) [2,2,1]$\\
$3) [3,3,2]$\\
$4) [4,4,2,5,1]$\\
$5) [2,5,1]$\\
$)[5,6]$\\


$1) 2^5*3^2*5^3*7^6*11^3 = 5637269484000 $\\
$2) 2^3*3^3*5^2 = 5400 $\\
$3) 2^4*3^4*5^3 = 162000$\\
$4) 2^5*3^5*5^3*7^6*11^2 = 13836934188000 $\\
$5) 2^3*3^6*5^2 = 145800$\\
$6) 2^6*3^7 = 139968$

\text{Код машины : } $2^{5637269484001} * 3^{5401} * 5^{162001} *7^{13836934188001} *11^{145801} * 13^{139969}$\\
\text{при x=1 :}\\
$1;1,0$\\
$2;1,1$\\
$3;0,1$\\
$4;0,1$\\
$5;0,2$\\
$6;0,2$\\

\text{Кодировка завершающегося вычисления: }\\

$[1,1,0] = 2^2*3^2*5^1 = 180$\\
$[2,1,1]= 2^3*3^2*5^2=1800$\\
$[3,0,1] = 2^4*3^1*5^2= 1200$\\
$[4,0,1] = 2^5*3^1*5^2= 2400$\\
$[5,0,2] = 2^6*3^1*5^3= 24000$\\
$[6,0,2] = 2^7*3^1*5^3= 48000$\\

\text{Код вычисления :} $2^{181} * 3^{1801} * 5^{1201} *7^{2401} *11^{24001} * 13^{48001}$

\item Упражнение 13 (полностью).\\
б)
$2^{308701} * 3^{26575698996001} * 5^{16201} * 7^{15553}$\\
\newline

$308700 = 2^2 * 3^2*5^2*7^3 = [1,1,1,3]$\\
$26575698996000 = 2^5*3^3*5^3*7^5*11^=[4,2,2,4,3]$\\
$16200 = 2^3*3^4*5^2 = [2,3,1]$\\
$15552 = 2^6*3^5 = [5,4]$\\
\newline
\text{Полученная регистровая машина:}
\begin{flalign*}
1&: x_1 \uleft 3&&\\
2&: \uif{x_{11}=0}{4}{3}&&\\
3&: x_{1}\uleft x_{1}+1&&\\
4&: \ustop&&
\end{flalign*}

\end{enumerate}

\subsection*{Упражнения по заданию №6}

\begin{enumerate}
\item Проблема соответствия Поста: любые четыре пункта.\\
	1) $A=(1,10111,10)$, $B=(111,10,0)$
	
	Допустим, что экземпляр ПСП имеет решение $i_{1},i_{2},...,i_{m}$, при некотором m.
	При $i_{1} = 3$ цепочка, начинающаяся с 10 должна равняться цепочке, начинающейся с 0. Но это равенство невозможно, поскольку их первые символы 0 и 1 , соответственно.\\
	Если $i_{1} = 1$, то две соответствующие цепочки из списков А и В должны начинаться так :
	\newline
	А : 1... \\
	В : 111...
	\newline
	Рассмотрим теперь, каким может быть $i_2$. 
	
	1. Вариант $i_2 = 2$ невозможен, поскольку никакая цепочка, начинающаяся с 110111 не может соответствовать цепочке,которая начинается с 11110; эти цепочки различаются в 3 позиции.
	\newline
	2.Вариант $i_2 = 3$ также невозможен, поскольку никакая цепочка, начинающаяся с 110 не может соответствовать цепочке,которая начинается с 1110; эти цепочки различаются в 3 позиции.
	\newline
	3. Возможен лишь вариант $i_2 = 1$.\\
	При $i_2 = 1$ цепочки имеют следующий вид :
	\newline
	А : 11...\\
	В : 111111...
	\newline
	Последовательность нельзя продолжить до решения, так как цепочка из списка В отличается от цепочки из списка А лишнимим символами 1 на конце.Чтобы избежать несовпадения, мы вынуждены выбирать $i_3 = 1$,$i_4 = 1$ и так далее. Таким образом, цепочка из списка А никогда не догонит цепочку из списка В, и решение никогда не будет получено.\\
	
	Если $i_{1} = 2$, то две соответствующие цепочки из списков А и В должны начинаться так :
	\newline
	А : 10111... \\
	В : 10...
	\newline
	Рассмотрим теперь, каким может быть $i_2$. 
	
	1. Вариант $i_2 = 2$ невозможен, поскольку никакая цепочка, начинающаяся с 1011110111 не может соответствовать цепочке,которая начинается с 1010; эти цепочки различаются в 4 позиции.
	\newline
	2.Вариант $i_2 = 3$ также невозможен, поскольку никакая цепочка, начинающаяся с 1011110 не может соответствовать цепочке,которая начинается с 100; эти цепочки различаются в 3 позиции.
	\newline
	3. Возможен лишь вариант $i_2 = 1$.\\
	При $i_2 = 1$ цепочки имеют следующий вид :
	\newline
	А : 101111...\\
	В : 10111...
	\newline
	Последовательность нельзя продолжить до решения, так как цепочка из списка А отличается от цепочки из списка В одним лишнимим символом 1 на конце.Чтобы избежать несовпадения, мы вынуждены выбирать $i_3 = 1$,$i_4 = 1$ и так далее. Таким образом, цепочка из списка В никогда не догонит цепочку из списка А, и решение никогда не будет получено.\\
	\newline
	Вывод : данный экземпляр ПСП не имеет решения.\\
	
	2)  $A=(10,011,101)$, $B=(101,11,011)$
	
	Допустим, что экземпляр ПСП имеет решение $i_{1},i_{2},...,i_{m}$, при некотором m.
	При $i_{1} = 2$ цепочка, начинающаяся с 011 должна равняться цепочке, начинающейся с 11. Но это равенство невозможно, поскольку их первые символы 0 и 1 , соответственно.\\
	При $i_{1} = 3$ цепочка, начинающаяся с 101 должна равняться цепочке, начинающейся с 011. Но это равенство невозможно, поскольку их первые символы 0 и 1 , соответственно.\\
	Если $i_{1} = 1$, то две соответствующие цепочки из списков А и В должны начинаться так :
	\newline
	А : 10... \\
	В : 101...
	\newline
	Рассмотрим теперь, каким может быть $i_2$. 
	
	1. Вариант $i_2 = 1$ невозможен, поскольку никакая цепочка, начинающаяся с 1010 не может соответствовать цепочке,которая начинается с 101101; эти цепочки различаются в 4 позиции.
	\newline
	2.Вариант $i_2 = 2$ также невозможен, поскольку никакая цепочка, начинающаяся с 10011 не может соответствовать цепочке,которая начинается с 10111; эти цепочки различаются в 3 позиции.
	\newline
	3. Возможен лишь вариант $i_2 = 3$.\\
	При $i_2 = 3$ цепочки имеют следующий вид :
	\newline
	А : 10101...\\
	В : 101011...
	\newline
	Последовательность нельзя продолжить до решения, так как цепочка из списка В отличается от цепочки из списка А одним лишнимим символом 1 на конце.Чтобы избежать несовпадения, мы вынуждены выбирать $i_3 = 3$,$i_4 = 3$ и так далее. Таким образом, цепочка из списка А никогда не догонит цепочку из списка В, и решение никогда не будет получено.\\
	\newline
	Вывод : данный экземпляр ПСП не имеет решения.\\
	
	3) $A=(01,001,10)$, $B=(011,10,00)$
	
	Допустим, что экземпляр ПСП имеет решение $i_{1},i_{2},...,i_{m}$, при некотором m.
	При $i_{1} = 2$ цепочка, начинающаяся с 001 должна равняться цепочке, начинающейся с 10. Но это равенство невозможно, поскольку их первые символы 0 и 1 , соответственно.\\
	При $i_{1} = 3$ цепочка, начинающаяся с 10 должна равняться цепочке, начинающейся с 00. Но это равенство невозможно, поскольку их первые символы 0 и 1 , соответственно.\\
	Если $i_{1} = 1$, то две соответствующие цепочки из списков А и В должны начинаться так :
	\newline
	А : 01... \\
	В : 011...
	\newline
	Рассмотрим теперь, каким может быть $i_2$. 
	
	1. Вариант $i_2 = 1$ невозможен, поскольку никакая цепочка, начинающаяся с 0101 не может соответствовать цепочке,которая начинается с 011011; эти цепочки различаются в 3 позиции.
	\newline
	2.Вариант $i_2 = 2$ также невозможен, поскольку никакая цепочка, начинающаяся с 01001 не может соответствовать цепочке,которая начинается с 01110; эти цепочки различаются в 3 позиции.
	\newline
	3. Возможен лишь вариант $i_2 = 3$.\\
	При $i_2 = 3$ цепочки имеют следующий вид :
	\newline
	А : 0110...\\
	В : 01100...
	\newline
	Рассмотрим , каким может быть $i_3$. 
	
	1. Вариант $i_3 = 1$ невозможен, поскольку никакая цепочка, начинающаяся с 011001 не может соответствовать цепочке,которая начинается с 01100011; эти цепочки различаются в 6 позиции.
	\newline
	2.Вариант $i_3 = 2$ также невозможен, поскольку никакая цепочка, начинающаяся с 0110001 не может соответствовать цепочке,которая начинается с 0110010; эти цепочки различаются в 6 позиции.
	\newline
	3. .Вариант $i_3 = 3$ также невозможен, поскольку никакая цепочка, начинающаяся с 011010 не может соответствовать цепочке,которая начинается с 0110000; эти цепочки различаются в 5 позиции.\\
	\newline
	Вывод : данный экземпляр ПСП не имеет решения.\\
	
	4) $A=(01,001,10)$, $B=(011,01,00)$
	
	
	При $i_{1} = 2$ цепочка, начинающаяся с 001 должна равняться цепочке, начинающейся с 01. Но это равенство невозможно, поскольку их вторые символы 0 и 1 , соответственно.\\
	При $i_{1} = 3$ цепочка, начинающаяся с 10 должна равняться цепочке, начинающейся с 00. Но это равенство невозможно, поскольку их первые символы 0 и 1 , соответственно.\\
	Если $i_{1} = 1$, то две соответствующие цепочки из списков А и В должны начинаться так :
	\newline
	А : 01... \\
	В : 011...
	\newline
	Рассмотрим теперь, каким может быть $i_2$. 
	
	1. Вариант $i_2 = 1$ невозможен, поскольку никакая цепочка, начинающаяся с 0101 не может соответствовать цепочке,которая начинается с 011011; эти цепочки различаются в 3 позиции.
	\newline
	2.Вариант $i_2 = 2$ также невозможен, поскольку никакая цепочка, начинающаяся с 01001 не может соответствовать цепочке,которая начинается с 01101; эти цепочки различаются в 3 позиции.
	\newline
	3. Возможен лишь вариант $i_2 = 3$.\\
	При $i_2 = 3$ цепочки имеют следующий вид :
	\newline
	А : 0110...\\
	В : 01100...
	\newline
	Рассмотрим , каким может быть $i_3$. 
	
	1. Вариант $i_3 = 1$ невозможен, поскольку никакая цепочка, начинающаяся с 011001 не может соответствовать цепочке,которая начинается с 0110001; эти цепочки различаются в 6 позиции.
	\newline
	2.Вариант $i_3 = 3$ также невозможен, поскольку никакая цепочка, начинающаяся с 0110010 не может соответствовать цепочке,которая начинается с 0110000; эти цепочки различаются в 6 позиции.
	\newline
	3. При $i_3 = 2$ цепочки имеют следующий вид :
	\newline
	А : 0110001\\
	В : 0110001
	\newline
	Вывод : полученные цепочки совпадают,данный экземпляр ПСП имеет решение 1,3,2.\\
	
\item Классы P и NP:
  \begin{itemize}
  \item упражнение 1 (любые два пункта);\\
  
  \item упражнение 3 (любой пункт);\\
  
  \end{itemize}

\item Булевы формулы:
  \begin{itemize}
  \item упражнение 1 (любые четыре пункта);\\
  a) $xy+\overline{x}z = (x \wedge y)\vee(\overline{x} \wedge z)$
  
  \begin{center}
  	\begin{tikzpicture}[-,>=stealth',level/.style={sibling distance = 3cm/#1,level distance = 1.2cm}]
  	
  	\node [app] {$\vee$}
  	child{ node [app] {$\wedge$}
  		child{node [var] {$x$}}
  		child{node [var] {$y$}}
  	}                          
  	child{ node [app] {$\wedge$}
  		child{node [var] {$\overline{x}$}}
  		child{node [var] {$z$}}
  	}           
  	; 
  	\end{tikzpicture}
  \end{center}
  
  КНФ : $(x \vee t)(y\vee t)(\overline{x} \vee \overline{t})(z \vee \overline{t}) $\\
  3КНФ : $(x \vee t \vee v)(x \vee t \vee \overline{v})(y \vee t \vee p)(y \vee t \vee \overline{p})(\overline{x} \vee \overline{t} \vee w)(\overline{x} \vee \overline{t} \vee \overline{w})(z \vee \overline{t} \vee q)(z \vee \overline{t} \vee \overline{q})$\\
  \newline
  Удовлетворяющая подстановка для формулы : $x = 1, y =1, z=0$\\
  Удовлетворяющая подстановка для 3КНФ : $x = 1, y =1, z=0, t=0, w=0,p=0,w=0,q=0$\\
  
  б) $wxyz+u+v = (w \wedge x)\wedge(y \wedge z) \vee (u \vee v)$
  
  \begin{center}
  	\begin{tikzpicture}[-,>=stealth',level/.style={sibling distance = 3cm/#1,level distance = 1.2cm}]
  	
  	\node [app] {$\vee$}
  	child{ node [app] {$\wedge$}
  		child{node [app] {$\wedge$}
  			child{node [var] {$w$}}
  			child{node [var] {$x$}}
  		}
  		child{node [app] {$\wedge$}
  			child{node [var] {$y$}}
  			child{node [var] {$z$}}
  		}
  	}                          
  	child{ node [app] {$\vee$}
  		child{node [var] {$u$}}
  		child{node [var] {$v$}}
  	}           
  	; 
  	\end{tikzpicture}
  \end{center}
  
  КНФ : $(w \vee p)(x\vee p)(y \vee p)(z \vee p)(u  \vee t \vee \overline{p})(v \vee \overline{t} \vee \overline{p}) $\\
  3КНФ : $(w \vee p \vee q)(w \vee p \vee \overline{q})(x \vee p \vee a)(x \vee p \vee \overline{a})(y \vee p \vee b)(y \ vee p \vee \overline{b})(z \vee p \vee c)(z \vee p \vee \overline{c})(u  \vee t \vee \overline{p})(v \vee \overline{t} \vee \overline{p})$\\
  \newline
  Удовлетворяющая подстановка для формулы : $w =1, x = 1, y =1, z=0, u=1,v=1$\\
  Удовлетворяющая подстановка для 3КНФ : $w =1, x = 1, y =1, z=0, u=1,v=1 , p=1,$ ост. = 0 \\
  
  в) $wxy+\overline x uv = (w \wedge (x \wedge y)) \vee ((\overline{x} \wedge u) \wedge v)$
  
  \begin{center}
  	\begin{tikzpicture}[-,>=stealth',level/.style={sibling distance = 3cm/#1,level distance = 1.2cm}]
  	
  	\node [app] {$\vee$}
  	child{ node [app] {$\wedge$}
  		child{node [var] {$w$}}
  		child{node [app] {$\wedge$}
  			child{node [var] {$x$}}
  			child{node [var] {$y$}}
  		}
  	}                          
  	child{ node [app] {$\wedge$}
  		child{node [app] {$\wedge$}
  			child{node [var] {$\overline{x}$}}
  			child{node [var] {$u$}}
  		}
  		child{node [var] {$v$}}
  	}           
  	; 
  	\end{tikzpicture}
  \end{center}
  
  КНФ : $(w \vee p)(x\vee p)(y \vee p)(\overline{x} \vee \overline{p})(u  \vee \overline{p})(v \vee \overline{p}) $\\
  3КНФ : $(w \vee p \vee t)(w \vee p \vee \overline{t})(x \vee p \vee q)(x \vee p \vee \overline{q})(y \vee p \vee z)(y \ vee p \vee \overline{z})(\overline{x} \vee \overline{p} \vee a)(\overline{x} \vee \overline{p} \vee \overline{a})(u  \vee \overline{p} \vee b)(u  \vee \overline{p} \vee \overline{b})(v \vee \overline{p} \vee c)(v \vee \overline{p} \vee \overline{c})$\\
  \newline
  Удовлетворяющая подстановка для формулы : $w =1, x = 1, y =1, u=1,v=1$\\
  Удовлетворяющая подстановка для 3КНФ : $w =1, x = 1, y =1, u=1,v=1,p=1,$ ост. = 1\\
  
  г) $x+\neg (\overline y + z) +t = (x \vee (y \wedge \overline{z})) \ vee t$
  
  \begin{center}
  	\begin{tikzpicture}[-,>=stealth',level/.style={sibling distance = 3cm/#1,level distance = 1.2cm}]
  	
  	\node [app] {$\vee$}
  	child{ node [app] {$\vee$}
  		child{node [var] {$x$}}
  		child{node [app] {$\wedge$}
  			child{node [var] {$y$}}
  			child{node [var] {$\overline{z}$}}
  		}
  	}                          
  	child{node [var] {$t$}}          
  	; 
  	\end{tikzpicture}
  \end{center}
  
  КНФ : $(x \vee p \vee q)(y \vee \overline{p} \vee q)(\overline{z} \vee \overline{p} \vee q)(t \vee \overline{q}) $\\
  3КНФ : $(x \vee p \vee q)(y \vee \overline{p} \vee q)(\overline{z} \vee \overline{p} \vee q)(t \vee \overline{q} \vee v)(t \vee \overline{q} \vee \overline{v})$\\
  \newline
  Удовлетворяющая подстановка для формулы : $ x = 1,$ ост. = 0\\
  Удовлетворяющая подстановка для 3КНФ : $x = 1, q =1, t = 1,$ ост. = 1\\
  
  \item упражнение 2 (любые четыре пункта).\\
  a) $x\wedge(\bar x\vee y)$\\
  3-КНФ: $(x \vee a \vee \bar b)  \wedge (x \vee a \vee b) \wedge  (x \vee \bar a \vee  \bar b) \wedge(\bar x\vee y \vee c)\wedge (x \vee\bar a \vee  b) \wedge (\bar x\vee y \vee \bar c)$\\
  Подстановка : y = 1, x = 1, а = 0 , b = 0 , c = 0 ;\\
  
  б) $\bar x \vee \bar y \vee z \vee \bar u$\\
  3-КНФ: $( x\vee \bar y \vee a) \wedge (z \vee \bar u \vee \bar a)$\\
  Подстановка: x = 1, z = 1, a = 0;\\
  
  в) $(x\vee \bar u) \wedge (\bar y \vee z)$\\
  3-КНФ: $( x\vee \bar u \vee a) \wedge ( x\vee \bar u \vee \bar a) \wedge (\bar y \vee z \vee b) \wedge (\bar y \vee z \vee \bar b)$\\
  Подстановка: x = 1, y = 0, a = 0 , b = 0;\\
  
  г) $(\bar x \vee \bar s) \wedge (x \vee z \vee y \vee s)$\\
  3-КНФ: $ (\bar x \vee \bar s \vee \bar a) \wedge (\bar x \vee \bar s \vee a) \wedge (x \vee z \vee b) \wedge (y \vee s \vee \bar b)$\\
  Подстановка: x = 0, z = 1, s = 1, a = 0 , b = 0, y = 0.\\
  
  \end{itemize}
\end{enumerate}


\end{document}
