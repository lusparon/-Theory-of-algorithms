\documentclass[12pt,a4paper]{article}

\usepackage[T2A]{fontenc}
\usepackage[utf8x]{inputenc}
\usepackage{indentfirst}
\usepackage{amssymb}
\usepackage{amsfonts}
\usepackage{amsmath}
\usepackage{tikz}
\usepackage{cmap}
\usepackage{euscript}
\usepackage{fancyhdr}
\usepackage{enumitem}
\usepackage{nicefrac}
\usepackage{multicol}
\usepackage{stmaryrd}
\usepackage[english,russian]{babel}
\usepackage{multicol}
\usepackage[center]{titlesec}
\usetikzlibrary{arrows,automata}

\renewcommand{\theenumii}{\asbuk{enumii}}
\AddEnumerateCounter{\asbuk}{\@asbuk}{\cyrm}


\textwidth=17cm
\hoffset=-0.8cm
\voffset=-1.2cm
%\headsep=0cm
\topmargin=0cm
\textheight=24cm
\oddsidemargin=0pt
\parindent=1.25cm



\linespread{1.06}

%\pagestyle{empty}
\pagestyle{fancy}
\lhead{\bfseries Теория алгоритмов}
\rhead{\itshape 2018/2019 (весенний семестр)}
%\fancyfoot{}

\renewcommand{\theenumii}{\asbuk{enumii}}

\newcommand{\FF}{\mathbb F}
\newcommand{\ZZ}{\mathbb Z}
\newcommand{\QQ}{\mathbb Q}
\newcommand{\RR}{\mathbb R}
\newcommand{\CC}{\mathbb C}

\newcommand{\tol}{\shortleftarrow}
\newcommand{\tor}{\shortrightarrow}
\newcommand{\tra}[3]{{\scriptsize $#1/#2\!#3$}}

\newcommand{\monus}{\stackrel{{}^{\scriptstyle .}}{\smash{-}}}
\begin{document}

\begin{center} 
\bfseries\LARGE Задание №6
\end{center}

\subsection*{Проблема соответствия Поста}


Найти решения следующих экземпляров проблемы соответствия Поста, 
либо доказать, что решения не существует:

\begin{multicols}{2}
  \begin{enumerate}[label=\arabic*),itemsep=5pt]
  \item $A=(1,10111,10)$, $B=(111,10,0)$;
  \item $A=(10,011,101)$, $B=(101,11,011)$;
  \item $A=(01,001,10)$, $B=(011,10,00)$;
  \item $A=(01,001,10)$, $B=(011,01,00)$;
  \item $A=(ab,a,bc,c)$, $B=(bc,ab,ca,a)$;
  \item $A=(abab,aaabbb,aab,ba,ab,aa)$,\\$B=(ababaaa,bb,baab,baa,ba,a)$.
  \end{enumerate}
\end{multicols}

\emph{Необходимые определения и примеры решения экземпляров проблемы соответствия Поста можно найти в~книге Хопкрофта, Мотвани и Ульмана «Введение в теорию автоматов, языков и вычислений» (глава 9, раздел 9.4).}

\subsection*{Классы  P и NP}

\begin{enumerate}[itemsep=5pt]
\item Докажите утверждения о замкнутости класса \textbf P относительно различных операций:
  \begin{enumerate}
  \item если $L_1, L_2\in  \mathbf P$ и $L=L_1\cup L_2$, то $L\in \mathbf P$ (объединение);
  \item если $L\in \mathbf P$, то $\overline L\in \mathbf P$ (дополнение);
  \item если $L_1, L_2\in  \mathbf P$ и $L=\{\alpha\beta: \alpha\in L_1, \beta\in L_2\}$, то $L\in \mathbf P$ (конкатенация).
  \end{enumerate}

\item Сформулируйте и докажите утверждения о том, что класс \textbf{NP} замкнут относительно  объединения и конкатенации. Что будет происходить в случае дополнения?
\item Докажите, что следующие задачи разрешения принадлежат классу \textbf{NP}:
  \begin{enumerate}[topsep=0mm,itemsep=0mm]
  \item раскраска вершин графа двумя цветами таким образом, чтобы 
любые две смежные вершины были разноцветными;
\item  раскраска вершин графа тремя цветами таким образом, чтобы 
любые две смежные вершины были разноцветными.
  \end{enumerate}
\end{enumerate}

\subsection*{Булевы формулы}

\begin{enumerate}[itemsep=5pt]
\item Постройте по следующим булевым формулам формулы в КНФ, а затем в 3-КНФ, выполнимые одновременно с исходными:
  \begin{multicols}{2}
  \begin{enumerate}
  \item $xy+\overline{x}z$;
  \item $wxyz+u+v$;
  \item $wxy+\overline x uv$;
  \item $x+\neg (\overline y + z) +t$;
  \item $(x+ yz)\neg(t+x\overline u)$;
  \item $\neg(x+y)\neg(x+z)\neg(y+z)$.
  \end{enumerate}
  \end{multicols}
Если исходная формула выполнима, постройте удовлетворяющую ей подстановку и её расширение на построенную формулу в 3-КНФ.
  
\item Постройте по следующим формулам в КНФ формулы в 3-КНФ, выполнимые одновременно с исходными,
и укажите удовлетворяющие им подстановки:
\begin{multicols}{2}
\begin{enumerate}[topsep=0mm]
\item $x\wedge(\bar x\vee y)$;
\item $\bar x\vee \bar y \vee z \vee \bar u$;
\item $(x\vee \bar u)\wedge(\bar y\vee z)$;
\item $(\bar x\vee \bar s)\wedge(x\vee z\vee y\vee s)$;
\item $x\vee y \vee z \vee u \vee v$;
\item $x\wedge (y\vee \bar z) \wedge (\bar x\vee s \vee \bar z \vee u)$.
\end{enumerate}
\end{multicols}
\vspace{-3mm}

\item Опишите работающий полиномиальное время алгоритм, вычисляющий
значение булевой формулы на заданной подстановке.

\item Посчитайте, сколько существует различных дизъюнктов с тремя литералами или их отрицаниями
в случае, если всего имеется $n$ переменных. Сколько существует формул в 3-КНФ с $m$ дизъюнктами
и $n$ переменными? Сколько существует формул в 3-КНФ с не более чем $m$ дизъюнктами
и $n$ переменными? 

\item Опишите работающий полиномиальное время алгоритм, выполняющий проверку выполнимости
булевой формулы в форме 2SAT. \textit{Подсказка}: предположить истинность 
одной из переменных и попытаться извлечь из этого предположения все возможные следствия.

\item Докажите, что проблема существования у булевой формулы как минимум четырёх
различных удовлетворяющих ей подстановок является NP-полной.

\end{enumerate}
\end{document}
