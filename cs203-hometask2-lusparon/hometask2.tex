\documentclass[12pt,a4paper]{article}
\usepackage[T2A]{fontenc}
\usepackage[utf8]{inputenc}
\usepackage[english,russian]{babel}
\usepackage{amssymb}
\usepackage{amsfonts}
\usepackage{amsmath}
\usepackage{cmap}
\usepackage{indentfirst}
\usepackage{fancyhdr}
\usepackage{stmaryrd}
\usepackage{enumitem}
\usepackage{multicol}
\usepackage{url}
\usepackage[center]{titlesec}
\usepackage{tikz}
\usetikzlibrary{arrows,positioning,shapes,automata} 
\tikzset{
	treenode/.style = {align=center, inner sep=0pt, text centered,
		font=\sffamily,minimum height=1.2em},
	app/.style = {treenode, black, rectangle, font=\itshape,
		text width=2.2em},% arbre rouge noir, noeud noir
	lam/.style = {treenode,  black,  
		text width=1.5em},% arbre rouge noir, noeud rouge
	var/.style = {treenode, circle, 
		minimum width=1.5em, minimum height=0.5em}% arbre rouge noir, nil
}

\textwidth=17cm
\voffset=-1cm
\hoffset=-0.5cm
\topmargin=0cm
\textheight=24.5cm
\oddsidemargin=0pt

\pagestyle{fancy}
%%%%%%%%%%%%%%%%%%%%%%%%%%%%%%%%%%%%%%%%%%%%%%%%%%%%%%%%%%%%%%%%%%%%%%%%%%%%%%%%%%%%%%%%%%%%%%%%%%%%%%%%%

\lhead{\bfseries Домашнее задание №2}
\rhead{Тызыхян Луспарон, 2.8}

%%%%%%%%%%%%%%%%%%%%%%%%%%%%%%%%%%%%%%%%%%%%%%%%%%%%%%%%%%%%%%%%%%%%%%%%%%%%%%%%%%%%%%%%%%%%%%%%%%%%%%%%% 


\fancyfoot{}

\renewcommand{\L}{\ensuremath{\lambda}}
\newcommand{\Lvar}[1]{\ensuremath{\L #1\,.\,}}
\newcommand{\Lx}{\Lvar{x}}
\newcommand{\Ly}{\Lvar{y}}
\newcommand{\Lz}{\Lvar{z}}
\newcommand{\Lt}{\Lvar{t}}
\newcommand{\Ln}{\Lvar{n}}



% Команды для описания регистровых машин

\newcommand{\monus}{\stackrel{{}^{\scriptstyle .}}{\smash{-}}}
\newcommand{\recop}[2]{\textbf{#1}(#2)}
\newcommand{\Comp}[1]{\recop{comp}{#1}}
\newcommand{\Prr}[1]{\recop{prim}{#1}}
\newcommand{\uif}[3]{\mathrm{if}\ #1\  \mathrm{then\ goto}\ #2\ \mathrm{else\ goto}\ #3}
\newcommand{\ustop}{\mathrm{stop}}
\newcommand{\uleft}{\leftarrow}


\newcommand{\PR}{\mathcal{PR}}


\begin{document}
\section*{Домашнее задание №2}

\subsection*{Задание №1: $\lambda$-исчисление}


\subsubsection*{Задания по упражнению 1}

\emph{Выполните любые два пункта.}

Решение.\\ 
1д) 
$ux(yz)(\lambda v.vy) = (((ux)(yz))(\lambda v.(vy)))$
\begin{center}
\begin{tikzpicture}[-,>=stealth',level/.style={sibling distance = 3cm/#1,level distance = 1.2cm}]
	
\node [app] {App}
	child{ node [app] {App} 
		child{ node [app] {App}
			child{ node [app] {App}
				child{ node [var] {u}}
				child{ node [var] {x}}
			}
			child{ node [app] {App}
			    child{ node [var] {y}}
			    child{ node [var] {z}}
		    }
		}
	}                        
	child{ node [lam] {$\L v$}
		child{ node [app] {App}
			child{ node [var] {$v$}}
			child{ node [var] {$y$}} 
		}	
	}
;
\end{tikzpicture}
\end{center}
1з) $vw(\lambda xy.vx) = ((vw)(\lambda (xy).(vx)))$
\begin{center}
\begin{tikzpicture}[-,>=stealth',level/.style={sibling distance = 3cm/#1,level distance = 1.2cm}]
	
\node [app] {App}
	child{ node [app] {App} 
		child{ node [var] {$v$}}
		child{ node [var] {$w$}}
	}                        
	child{ node [lam] {$\L x$}
		child{ node [lam] {$\L y$}
			child{ node [app] {App}
				child{ node [var] {$v$}}
				child{ node [var] {$x$}}
			}
		}
	} 
;
\end{tikzpicture}
\end{center}

\subsubsection*{Задания по упражнению 2}

\emph{Выполните любые три пункта.}

Решение.\\
2а) $[uv/x](\Lx zy) = \Lx zy$\\
2д) $[uy/x](x\Ly yx) = \\= (uy)([uy/x](\Ly yx)) = \\= (uy)(\Lz [uy/x]([z/y](yx))) = \\= (uy)(\Lz z(uy)).$\\
2е) $[uv/x](\Ly x( \L w.vwx)) = \\= \Ly [uv/x](x(\L w.vwx)) = \\= \Ly (uv)([uv/x](\L w.vwx)) = \\= \Ly (uv)(\L w.[uv/x](vwx)) = \\= \Ly (uv)(\L w.vw(uv)).$



\subsubsection*{Задания по упражнению 3}

\emph{Выполните любые четыре пункта.}

Решение.
\begin{enumerate}
\item
(б)
\[
\begin{array}{ll}
\underline{(\L(xy).yx)\,u}v &\betared [u/x](\Ly yx)\,v = \underline{(\Ly yu)\,v}\\
&\betared [v/y](yu) = vu;\\
\end{array}
\]
\item
(г)
\[
\begin{array}{ll}
\underline{(\Lx xxy)\,(\Ly yz)} &\betared [\Ly yz/x](xxy) = \underline{(\Ly yz)(\Ly yz)}\,y\\
&\betared [\Ly yz](yz)\,y = \underline{(\Ly yz)\,z}\,y\\ &\betared [z/y](yz)y = zzy;\\
\end{array}
\]
\item
(з)
\[
\begin{array}{ll}
\underline{(\L(xy).yx)\,(uv)}\,zw &\betared [uv/x](\Ly yx)\,zw = (\Ly [uv/x](yx) )\,zw = \underline{(\Ly y(uv))z}\,w \\ &\betared [z/y](y(uv))w = z(uv)w;\\
\end{array}
\]
\item
(к)
\[
\begin{array}{ll}
\underline{(\L(xyz).xz(yz))(\L(uv).u)} &\betared [\L(uv).u/x](\L(yz).xz(yz)) = \L(yz).[\L(uv).u/x](xz(yz)) = \\ = \L(yz).\underline{(\L(uv).u)z}\,(yz) \\ &\betared \L(yz).[z/u](\L v.u)(yz) = \L(yz).\underline{(\L v.z)(yz)} \\ &\betared \L(yz).[yz/v](z) = \L(yz).z;\\
\end{array}
\]
\end{enumerate}


\subsubsection*{Задания по упражнению 4}

\emph{Выполните любые два пункта.}

Решение.
\begin{enumerate}
\item
(a) нормальный порядок
\[
\begin{array}{ll}
\underline{(\Lx x(x(yz))x)\,(\L u.uv)} &\betared [\L u.uv/x](x(x(yz))x) = \underline{(\L u.uv)\,((\L u.uv)(yz))}\,(\L u.uv) \\ &\betared [(\L u.uv)(yz)/u](uv)(\L u.uv) = (\underline{(\L u.uv)(yz)}v)(\L u.uv) \\ &\betared ([yz/u](uv)v)(\L u.uv) = yzvv\L u.uv ;\\
\end{array}
\]
\item
(a) аппликативный порядок
\[
\begin{array}{ll}
\underline{(\Lx x(x(yz))x)\,(\L u.uv)} &\betared [\L u.uv/x](x(x(yz))x) = (\L u.uv)\,\underline{((\L u.uv)(yz))}\,(\L u.uv) \\ &\betared (\L u.uv)([yz/u](uv))(\L u.uv) = \underline{(\L u.uv)(yzv)}(\L u.uv) \\ &\betared [yzv/u](uv)(\L u.uv) = yzvv\L u.uv ;\\
\end{array}
\]
\item
(в) нормальный порядок
\[
\begin{array}{ll}
\underline{(\Lx y)((\Ly yy)(\Ly yy))} &\betared [(\Ly yy)(\Ly yy)/x](y) = y;\\
\end{array}
\]
\item
(в) аппликативный порядок
\[
\begin{array}{ll}
(\Lx y)\underline{((\Ly yy)(\Ly yy))} &\betared (\Lx y)([\Ly yy/y](yy)) = (\Lx y)((\Ly yy)(\Ly yy)) ...\\
\end{array}
\]
\end{enumerate}

\subsubsection*{Задания по упражнению 5}

\emph{Выполните любые два пункта.}

Решение.

\subsubsection*{Задания по упражнению 7}

\emph{Выполните любые два пункта.}

Решение.
\begin{enumerate}
\item 
(a)
\[
\begin{array}{lcl}
T[\Lx y] &=&\\
&=& \cK ( T[y])\quad \text{(правило 4)}\\
&=& \cK y\quad \text{(правило 1)}\\
\end{array}
\]
\item 
(б)
\[
\begin{array}{lcl}
T[\Ly\Lx xx] &=&\\
&=& \cK (T[\Lx xx])\quad \text{(правило 4)}\\
&=& \cK (\cS (T[\Lx x])(\Lx x))\quad \text{(правило 6)}\\
&=& \cK (\cS \cI \cI)\quad \text{(правило 3)}\\
\end{array}
\]
\end{enumerate}

\end{document}
