\documentclass[12pt,a4paper]{article}
\usepackage[T2A]{fontenc}
\usepackage[utf8]{inputenc}
\usepackage[english,russian]{babel}
\usepackage{amssymb}
\usepackage{amsfonts}
\usepackage{amsmath}
\usepackage{cmap}
\usepackage{indentfirst}
\usepackage{fancyhdr}
\usepackage{stmaryrd}
\usepackage{enumitem}
\usepackage{multicol}
\usepackage[center]{titlesec}
\usepackage{tikz}
\usetikzlibrary{arrows,positioning,shapes,automata} 
\tikzset{
  treenode/.style = {align=center, inner sep=0pt, text centered,
    font=\sffamily,minimum height=1.2em},
  app/.style = {treenode, black, rectangle, font=\itshape,
     text width=2.2em},% arbre rouge noir, noeud noir
  lam/.style = {treenode,  black,  
    text width=1.5em},% arbre rouge noir, noeud rouge
  var/.style = {treenode, circle, 
    minimum width=1.5em, minimum height=0.5em}% arbre rouge noir, nil
}




\textwidth=17cm
\voffset=-1cm
\hoffset=-0.5cm
\topmargin=0cm
\textheight=24.5cm
\oddsidemargin=0pt

\renewcommand{\L}{\ensuremath{\lambda}}
\newcommand{\Lvar}[1]{\ensuremath{\L #1\,.\,}}
\newcommand{\Lx}{\Lvar{x}}
\newcommand{\Ly}{\Lvar{y}}
\newcommand{\Lz}{\Lvar{z}}
\newcommand{\Lt}{\Lvar{t}}
\newcommand{\Ln}{\Lvar{n}}

\newcommand{\betared}{\rightarrow_\beta}
\newcommand{\betareds}{\rightarrow_\beta^*}

\newcommand{\Lif}[3]{\text{if } #1 \text{ then } #2 \text{ else } #3}
%\newcommand{\Lif}[3]{\mathrm{if}\ #1\ \mathrm{then}\ #2\ \mathrm{else}\ #3}
\newcommand{\letin}[3]{\mathrm{let}\ #1 = #2\ \mathrm{in}\ #3}

\newcommand{\isZero}[1]{\text{\ttfamily is\_zero? } #1}
\newcommand{\Lexp}{\text{\ttfamily{} exp }}
\newcommand{\Lplus}{\text{\ttfamily plus }}
\newcommand{\Lsucc}{\text{\ttfamily succ }}
\newcommand{\Lpred}{\text{\ttfamily pred }}
\newcommand{\LNat}{\text{Nat}}
\newcommand{\LBool}{\text{Bool}}

\newcommand{\false}{\text{\ttfamily false }}
\newcommand{\true}{\text{\ttfamily true }}
\newcommand{\Land}{\text{\ttfamily{} and }}
\newcommand{\Lor}{\text{\ttfamily{} or }}
\newcommand{\Lnot}{\text{\ttfamily not }}

\newcommand{\fst}{\text{\ttfamily fst }}
\newcommand{\snd}{\text{\ttfamily snd }}

\DeclareMathOperator{\FV}{FV}

\newcommand{\Ycr}{\mathbf Y_{\text{CR}}}
\newcommand{\Yt}{\mathbf Y_{\text{Turing}}}

\newcommand{\clS}{\mathbf S}
\newcommand{\clK}{\mathbf K}
\newcommand{\clB}{\mathbf B}
\newcommand{\clI}{\mathbf I}
\newcommand{\clW}{\mathbf W}
\newcommand{\clC}{\mathbf C}
\newcommand{\clY}{\mathbf Y}
\newcommand{\CLvar}[1]{\ensuremath{[#1] \, . \,}}

\newcommand{\cI}{\textbf{I}}
\newcommand{\cK}{\textbf{K}}
\newcommand{\cS}{\textbf{S}}




\begin{document}
\section*{Пример дерева разбора $\lambda$-терма}

\begin{center}
\begin{tikzpicture}[-,>=stealth',level/.style={sibling distance = 3cm/#1,level distance = 1.2cm}]
    
\node [app] {App}
    child{ node [lam] {$\L x$} 
            child{ node [lam] {$\L y$}
                  child{ node [app] {App}
							child{ node [var] {$x$}}
							child{ node [var] {$y$}}
                  }
            }                            
    }
    child{ node [lam] {$\L x$}
            child{ node [var] {$v$} 
            }
		}
; 
\end{tikzpicture}
\end{center}

\section*{Примеры вычисления нормальной формы}
\begin{enumerate}
\item 
\[
\begin{array}{ll}
  (\Lx \underline{(\Ly yx)\,z})\,v &\betared (\Lx [z/y](yx))\,v = (\Lx zx)\,v\\
  	&\betared [v/x](zx)=zv;\\
 \underline{(\Lx (\Ly yx)\,z)\,v} &\betared [v/x]((\Ly yx)\,z) = (\Ly yv)\,z\\
  	&\betared [z/y](yv)=zv.
\end{array}
\]
\item 
\[
\begin{array}{lcl}
  \mathrm{is\_zero?}\ 0 &=& (\Lvar{n} n\,(\Lx false)\,true)\,(\Lx\Ly y)\\
  	&=& (\Lx\Ly y) (\Lx false)\,true\\
  	&=& (\Ly y)\, true\\
  	&=& true;\\
  \mathrm{is\_zero?}\ 1 &=& (\Lvar{n} n\,(\Lx false)\,true)\,(\Lx\Ly xy)\\
    &=& (\Lx\Ly xy) (\Lx false)\,true\\
  	&=& (\Ly (\Lx false)\,y)\, true\\
  	&=& (\Lx false)\, true\\
  	&=& false.  	
\end{array}
\]
\end{enumerate}

\section*{Пример преобразования в комбинаторную форму}

\[
\begin{array}{lcl}
T[\Lx\Ly yx] &=&\\
&=& T[\Lx T[\Ly (y x)]]\quad \text{(правило 5)}\\
&=& T[\Lx (\cS T[\Ly y] T[\Ly x])]\quad \text{(правило 6)}\\
&=& T[\Lx (\cS \cI T[\Ly x])]\quad \text{(правило 4)}\\
&=& T[\Lx (\cS \cI (\cK x))]\quad \text{(правила 3 и 1)}\\
&=& (\cS T[\Lx (\cS \cI)] T[\Lx (\cK x)])\quad \text{(правило 6)}\\
&=& (\cS (\cK (\cS \cI)) T[\Lx (\cK x)])\quad \text{(правило 3)}\\
&=& (\cS (\cK (\cS \cI)) (\cS T[\Lx \cK] T[\Lx x]))\quad \text{(правило 6)}\\
&=& (\cS (\cK (\cS \cI)) (\cS (\cK \cK) T[\Lx x]))\quad \text{(правило 3)}\\
&=& (\cS (\cK (\cS \cI)) (\cS (\cK \cK) \cI))\quad \text{(правило 4)}\\
\end{array}
\]

\end{document}
