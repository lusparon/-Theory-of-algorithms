\documentclass[12pt,a4paper]{article}
\usepackage[T2A]{fontenc}
\usepackage[utf8]{inputenc}
\usepackage[english,russian]{babel}
\usepackage{amssymb}
\usepackage{amsfonts}
\usepackage{amsmath}
\usepackage{cmap}
\usepackage{indentfirst}
\usepackage{fancyhdr}
\usepackage{stmaryrd}
\usepackage{enumitem}
\usepackage{multicol}
\usepackage[center]{titlesec}
\usepackage{tikz}
\usetikzlibrary{arrows,positioning,shapes,automata} 
\tikzset{
  treenode/.style = {align=center, inner sep=0pt, text centered,
    font=\sffamily,minimum height=1.2em},
  app/.style = {treenode, black, rectangle, font=\itshape,
     text width=2.2em},% arbre rouge noir, noeud noir
  lam/.style = {treenode,  black,  
    text width=1.5em},% arbre rouge noir, noeud rouge
  var/.style = {treenode, circle, 
    minimum width=1.5em, minimum height=0.5em}% arbre rouge noir, nil
}




\textwidth=17cm
\voffset=-1cm
\hoffset=-0.5cm
\topmargin=0cm
\textheight=24.5cm
\oddsidemargin=0pt


% Команды для описания диаграмм переходов
\newcommand{\tol}{\shortleftarrow}
\newcommand{\tor}{\shortrightarrow}
\newcommand{\tra}[3]{{\scriptsize $#1/#2\!#3$}}

\begin{document}


\subsection*{Пример задания функции переходов  МТ таблицей}

 Дана машина Тьюринга
$M=\left(\{q_0, q_1, q_2, q_3\},\{1,*\},\{1,*,B\},\delta,q_0,B,\{q_3\}\right)$, функция
переходов которой задана таблицей:
\begin{center}
  \begin{tabular}[t]{c||c|c|c}
     & $B$ & 1 & $*$\\
    \hline\hline
    $q_0$ & --- &$(q_1,B,R)$& $(q_3,B,R)$\\
    $q_1$ &$(q_2,1,L)$  &$(q_1,1,R)$& $(q_1,*,R)$\\
    $q_2$ &$(q_0,B,R)$  &$(q_2,1,L)$& $(q_2,*,L)$\\
    $q_3$ & --- & --- & ---
  \end{tabular}
  
\end{center}


\subsection*{Пример задания функции переходов  МТ диаграммой}

Дана машина Тьюринга 
$M=\left(\{q_0, q_1, q_2, q_3, q_4\},\{0, 1\},\{0,1,X,Y,B\},\delta,q_0,B,\{q_4\}\right)$,
функция переходов которой задана диаграммой:

\begin{center}
  \begin{tikzpicture}[->,>=stealth',shorten >=1pt,auto,node distance=2.8cm,semithick]
\tikzstyle{every node}=[font=\small]
  \node[initial, initial text=,state] (A) {$q_0$};
  \node[state] (B) [right of=A] {$q_1$};
  \node[state] (C) [right of=B] {$q_2$};
  \node[state] (D) [below of=A] {$q_3$};
  \node[state,accepting] (E) [right of=D] {$q_4$};

  \path (A) edge              node {\tra 0X\tor} (B)
        (B) edge [loop above] node {\tra YY\tor, \tra 00\tor} (B)
        (B) edge              node {\tra 1Y\tol} (C)
        (C) edge [loop right] node {\tra YY\tol, \tra 00\tol} (C)
        (C) edge [bend left]  node {\tra XX\tor} (A)
        (A) edge              node {\tra YY\tor} (D)
        (D) edge [loop below] node {\tra YY\tor} (D)
        (D) edge              node {\tra BB\tor} (E);
\end{tikzpicture}
\end{center}

\subsection*{Пример задания функции переходов МТ отдельными правилами}

Дана машина Тьюринга
$M=\left(\{q_0, q_1, q_2, q_f\},\{0,1\},\{0,1,B\},\delta,q_0,B,\{q_f\}\right)$,
функция переходов которой задана набором правил:
\begin{align*}
\delta(q_0,0)&=(q_0,1,R),\\
\delta(q_1,1)&=(q_2,0,L), \\
\delta(q_2,1)&=(q_0,1,R),\\
\delta(q_1,B)&=(q_f, B, R).
\end{align*}

\subsection*{Пример описания конфигураций МТ}

$q_00001 \vdash 0q_0001 \vdash 00q_001 \vdash 000q_01 \vdash 0001q_1B$.




\end{document}
