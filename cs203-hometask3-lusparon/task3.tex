\documentclass[12pt,a4paper]{article}

\usepackage[T2A]{fontenc}
\usepackage[utf8x]{inputenc}
\usepackage{indentfirst}
\usepackage{amssymb}
\usepackage{amsfonts}
\usepackage{amsmath}
\usepackage{tikz}
\usepackage{cmap}
\usepackage{euscript}
\usepackage{fancyhdr}
\usepackage{nicefrac}
\usepackage{multicol}
\usepackage{stmaryrd}
\usepackage[english,russian]{babel}
\usepackage{multicol}
\usepackage[center]{titlesec}
\usepackage{enumitem}
\usetikzlibrary{arrows,automata}

\textwidth=17cm
\hoffset=-0.8cm
%\voffset=0cm
%\voffset=-2cm
%\headsep=0cm
\topmargin=0cm
\textheight=23cm
\oddsidemargin=0pt
\parindent=1.25cm

\linespread{1.1}

\pagestyle{empty}
\pagestyle{fancy}
\lhead{\bfseries Теория алгоритмов}
\rhead{\itshape 2018/2019 (весенний семестр)}
%\fancyfoot{}


\renewcommand{\theenumii}{\asbuk{enumii}}
\AddEnumerateCounter{\asbuk}{\@asbuk}{\cyrm}


\newcommand{\FF}{\mathbb F}
\newcommand{\ZZ}{\mathbb Z}
\newcommand{\QQ}{\mathbb Q}
\newcommand{\RR}{\mathbb R}
\newcommand{\CC}{\mathbb C}

\newcommand{\tol}{\shortleftarrow}
\newcommand{\tor}{\shortrightarrow}
\newcommand{\tra}[3]{{\scriptsize $#1/#2\!#3$}}

\newcommand{\monus}{\stackrel{{}^{\scriptstyle .}}{\smash{-}}}
\begin{document}
\section*{Задание №3: машины Тьюринга}

\subsection*{Базовое определение, регистры и дорожки}

\begin{enumerate}[itemsep=0pt]
\item Дана машина Тьюринга
$M=\left(\{q_0, q_1, q_2, q_3\},\{1,*\},\{1,*,B\},\delta,q_0,B,\{q_3\}\right)$, функция
переходов $\delta$ которой задана таблицей:
\begin{center}
  \begin{tabular}[t]{c||c|c|c}
     & $B$ & 1 & $*$\\
    \hline\hline
    $q_0$ & --- &$(q_1,B,R)$& $(q_3,B,R)$\\
    $q_1$ &$(q_2,1,L)$  &$(q_1,1,R)$& $(q_1,*,R)$\\
    $q_2$ &$(q_0,B,R)$  &$(q_2,1,L)$& $(q_2,*,L)$\\
    $q_3$ & --- & --- & ---
  \end{tabular}
  
\end{center}

\begin{enumerate}[itemsep=0pt]
\item Нарисовать диаграмму переходов.
\item Выписать последовательности конфигураций для следующих слов:\\ 1) $1*11$; 2) $11*1$; 
3) $*111$; 4) $11*$. 
\item Установить закономерность в работе машины и описать задачи, 
решаемые каждым из состояний. 
\item Вычислить количество
операций до останова машины в зависимости от размера исходных данных.
\end{enumerate}

\item Функция переходов машины  
$M=\left(\{q_0, q_1, q_2, q_3, q_4\},\{0, 1\},\{0,1,X,Y,B\},\delta,q_0,B,\{q_4\}\right)$  задана
диаграммой:
\vspace{-3mm}
\begin{center}
  \begin{tikzpicture}[->,>=stealth',shorten >=1pt,auto,node distance=2.8cm,
                    semithick]
\tikzstyle{every node}=[font=\small]
  \node[initial, initial text=,state] (A)                    {$q_0$};
  \node[state]         (B) [right of=A] {$q_1$};
  \node[state]         (C) [right of=B] {$q_2$};
  \node[state]         (D) [below  of=A] {$q_3$};
  \node[state,accepting]         (E) [right of=D]       {$q_4$};

  \path (A) edge              node {\tra 0X\tor} (B)
        (B) edge [loop above] node {\tra YY\tor, \tra 00\tor} (B)
        (B) edge              node {\tra 1Y\tol} (C)
        (C) edge [loop right] node {\tra YY\tol, \tra 00\tol} (C)
        (C) edge [bend left]  node {\tra XX\tor} (A)
        (A) edge              node {\tra YY\tor} (D)
        (D) edge [loop below] node {\tra YY\tor} (D)
        (D) edge              node {\tra BB\tor} (E);
\end{tikzpicture}
\end{center}
\begin{enumerate}
\item Выписать последовательности конфигураций для следующих слов:\\ 1) $0011$; 2) $011$; 
3) $111$; 4) $01$. 
\item Сформулировать алгоритм, в соответствии с которым
эта машина действует. 
\item Описать язык этой машины (то есть множество всех слов, на которых машина 
завершается в допускающем состоянии) и определить, является ли он
рекурсивным (всегда ли эта машина завершается).
\end{enumerate}


\item Написать программы для машины Тьюринга, распознающие следующие языки:
\vspace{-3mm}
  \begin{enumerate}[itemsep=0pt]
  \item все слова над алфавитом $\Sigma=\{1\}$ с чётным количеством единиц;
  \item все слова над алфавитом $\Sigma=\{0,1\}$ с чётным количеством единиц;
  \item все слова над алфавитом $\Sigma=\{0,1\}$, которые заканчиваются на единицу;
  \item все слова над алфавитом $\Sigma=\{0,1\}$, которые содержат в точности один
нуль в~начале слова и произвольное количество единиц;
\vspace{-1mm}
\item все слова над алфавитом $\Sigma=\{0,1\}$, которые начинаются и заканчиваются
нулями и имеют произвольное количество единиц между ними.

  \end{enumerate}

\end{enumerate}


\subsection*{Разновидности машин Тьюринга}

\begin{enumerate}
\item Построить двухленточную машину Тьюринга, допускающую язык всех слов
  из~0 и~1, в которых этих символов поровну. Первая лента содержит вход и просматривается
  слева направо. Вторая лента используется для запоминания излишка нулей по отношению к 
  единицам и наоборот в прочитанной части входа. Описать состояния и переходы, а также
  неформальное предназначение каждого состояния. 

\item Описать идеи многоленточных машин Тьюринга, распознающих следующие языки:
  \begin{enumerate}
  \item  $L(M)=\{ww^R\mid w \mbox{\ --- произвольное слово из символов 0 и 1}\}$, символом
    $w^R$ обозначается слово, записанное теми же символами, что и $w$, но расположенными в обратном 
    порядке;
  \item $L(M)=\{a^nb^nc^n\mid n\geqslant 1\}$.
  \end{enumerate}

\item Машина Тьюринга, называемая $k$-головочной, имеет $k$ головок, читающих клетки на одной
ленте. Переход такой МТ зависит от состояния и символов, обозреваемых головками. За один переход
эта МТ может изменить состояние, записать новый символ в клетку, обозреваемую каждой из головок, 
и сдвинуть каждую головку влево, вправо или оставить на месте. Поскольку несколько головок могут
одновременно обозревать одну и ту же клетку, предполагается, что головки пронумерованы от~1
 до~$k$, и символ, который в действительности записывается в клетку, есть символ, записываемый
головкой с наибольшим номером. Доказать, что множества языков, допускаемых $k$-головочными
и обычными машинами Тьюринга, совпадают.

\item  Преобразовать машину Тьюринга
$M=\left(\{q_0, q_1, q_2\},\{0,1\},\{0,1,B\},\delta,q_0,B,\{q_2\}\right)$
к~виду, в котором машина не перемещается 
левее начального положения:
\begin{multicols}{2}
  \begin{enumerate}
  \item 
  \begin{tabular}[t]{c||c|c|c}
     $\delta$ & $B$ & 1 & $0$\\
    \hline\hline
    $q_0$ & --- &$(q_1,1,R)$& --- \\
    $q_1$ &$(q_2,0,L)$  & --- & --- \\
    $q_2$ & --- & --- & ---\\
  \end{tabular}

\item 
  \begin{tabular}[t]{c||c|c|c}
     $\delta$ & $B$ & 1 & $0$\\
    \hline\hline
    $q_0$ & --- &$(q_1,1,R)$& $(q_1,0,L)$\\
    $q_1$ &$(q_2,0,R)$  & --- & --- \\
    $q_2$ & --- & --- & ---\\
  \end{tabular}
  \end{enumerate}
\end{multicols}

Привести примеры работы полученной машины на нескольких словах.

\item Недетерминированная машина Тьюринга $M=\left(\{q_0, q_1, q_2\},\{0,1\},\{0,1,B\},\delta,q_0,B,\{q_2\}\right)$  представлена следующей функцией переходов:
\begin{center}
  \begin{tabular}[t]{c||c|c|c}
     & $0$ & 1 & $B$\\
    \hline\hline
    $q_0$ & $\{(q_0, 1, R)\}$ &$\{(q_1,0,R)\}$ & --- \\
    $q_1$ &$\{(q_1, 0, R), (q_0,0,L)\}$  &$\{(q_1,1,R),(q_0,1,L)\}$& $\{(q_2,B,R)\}$\\
    $q_2$ & --- & --- & ---
  \end{tabular}
  
\end{center}
Показать, какие конфигурации достижимы из начальной конфигурации, 
если входом является следующая цепочка:
\begin{multicols}{3}
\begin{enumerate}
\item $01$;
\item $001$;
\item $011$.
\end{enumerate}
\end{multicols}

\item Недетерминированная машина $M=\left(\{q_0, q_1, q_2, q_f\},\{0,1\},\{0,1,B\},\delta,q_0,B,\{q_f\}\right)$  представлена следующей функцией переходов:
  \begin{multicols}{2}
  \begin{itemize}
  \item  $\delta(q_0,0)=\{(q_0,1,R), (q_1,1,R)\}$,
  \item $\delta(q_1,1)=\{(q_2,0,L)\}$,
  \item $\delta(q_2,1)=\{(q_0,1,R)\}$,
  \item $\delta(q_1,B)=\{(q_f, B, R)\}$.
  \end{itemize}
    
  \end{multicols}
Описать смысл ветвления в первом правиле для функции $\delta$ и язык этой машины.

\item Описать идеи недетерминированных, возможно многоленточных, машин Тьюринга, распознающих
следующие языки:
\begin{enumerate}
\item язык всех слов из символов 0 и 1, содержащих некоторое подслово длиной 100, которое повторяется, причём необязательно подряд;
\item язык всех слов вида $w_1\#w_2\#\dots\#w_n$ для произвольного $n$, где подслова 
$w_i$ состоят из символов 0 или 1, причём для некоторого  $j$ подслово $w_j$ есть двоичная запись числа $j$;
\item язык того же вида, что и в пункте (б), в котором хотя бы для двух различных значений $j$ подслово $w_j$ есть двоичная запись числа $j$.
\end{enumerate}
\emph{Совет}: использование недетерминизма позволяет избежать итераций и сократить время 
работы в недетерминированном смысле, то есть лучше, чтобы ветвей было много, но они были 
короткими.




\end{enumerate}

\subsection*{Литература}
\begin{enumerate}
\item Хопкрофт Дж., Мотвани Р., Ульман Дж. Введение в теорию автоматов, языков и~вычислений. — 2-е изд. — М.: Вильямс, 2002. — 528 с. (Глава 8, разделы 8.2--8.4, 8.5.1.)
\end{enumerate}

\end{document}
