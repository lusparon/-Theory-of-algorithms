\documentclass[12pt,a4paper]{article}
\usepackage[T2A]{fontenc}
\usepackage[utf8]{inputenc}
\usepackage[english,russian]{babel}
\usepackage{amssymb}
\usepackage{amsfonts}
\usepackage{amsmath}
\usepackage{cmap}
\usepackage{indentfirst}
\usepackage{fancyhdr}
\usepackage{stmaryrd}
\usepackage{enumitem}
\usepackage{multicol}
\usepackage[center]{titlesec}
\usepackage{tikz}
\usetikzlibrary{arrows,positioning,shapes,automata} 

\textwidth=17cm
\voffset=-1cm
\hoffset=-0.5cm
\topmargin=0cm
\textheight=24.5cm
\oddsidemargin=0pt

\pagestyle{fancy}
%%%%%%%%%%%%%%%%%%%%%%%%%%%%%%%%%%%%%%%%%%%%%%%%%%%%%%%%%%%%%%%%%%%%%%%%%%%%%%%%%%%%%%%%%%%%%%%%%%%%%%%%%

\lhead{\bfseries Домашнее задание №1}
\rhead{Фамилия Имя, Курс.Группа}

%%%%%%%%%%%%%%%%%%%%%%%%%%%%%%%%%%%%%%%%%%%%%%%%%%%%%%%%%%%%%%%%%%%%%%%%%%%%%%%%%%%%%%%%%%%%%%%%%%%%%%%%% 


\fancyfoot{}


\begin{document}
\section*{Домашнее задание №3}


\begin{enumerate}
\item Раздел «Базовое определение, регистры и дорожки»:
  \begin{enumerate}
  \item упражнение 2;
  \item упражнение 3 (любые два пункта) — в решениях приветствуется использование регистров и многодорожечных лент.
  \end{enumerate}
\item Раздел «Разновидности машин Тьюринга»:
  \begin{enumerate}
  \item упражнение 1;
  \item упражнение 2 (один любой пункт);
  \item упражнение 4 (один любой пункт) — в решении необходимо пользоваться приёмом, использованным в доказательстве теоремы о существовании машины Тьюринга, не смещающейся левее начального положения;
  \item упражнение 5 (один любой пункт);
  \item упражнение 6;
  \item упражнение 7 (один любой пункт);
  \item \emph{необязательно:} упражнение 3 (за решение можно получить до 5 бонусных баллов).
  \end{enumerate}
\end{enumerate}

\end{document}
